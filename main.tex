\documentclass[12pt,twoside,a4paper]{book}

\usepackage[english]{babel}
\usepackage{geometry}
\usepackage{fontspec}
\usepackage{fancyhdr}
\usepackage{titlesec}
\usepackage{amsmath}
\usepackage{setspace}
\usepackage{tikz}
\usepackage{wrapfig}
\usepackage{graphicx}

\setmainfont[Ligatures=TeX]{Times New Roman}
\setlength{\headheight}{15pt}

\pagestyle{fancy}
\fancyhf{}
\fancyhead[LE,RO]{\thepage}
\fancyhead[C]{\leftmark}

\renewcommand{\chaptermark}[1]{\markboth{\MakeUppercase{#1}}{}}

\titleformat{\part}[display]{\normalfont\bfseries}{}{0pt}{\Huge}
\titleformat{\chapter}[display]{\normalfont\bfseries}{}{0pt}{\huge}
\titleformat{\section}[display]{\normalfont\bfseries}{}{0pt}{\LARGE}
\titleformat{\subsection}[display]{\normalfont\bfseries}{}{0pt}{\Large}
\titleformat{\subsubsection}[display]{\normalfont\bfseries}{}{0pt}{\large}

\begin{document}	
\title{Elements and practice of Naval Architecture}
\author{David Steel}
\date{1812}

\newgeometry{margin=1cm}
\begin{titlepage}
\doublespacing
\begin{center}

\footnotesize THE

\LARGE ELEMENTS AND PRACTICE

\tiny OF

\Huge \textbf{NAVAL ARCHITECTURE;}

\tiny OR,

\LARGE A Treatise

\tiny ON

\huge \textbf{SHIP-BUILDING,}

\small THEORETICAL AND PRACTICAL,

\large ON THE BEST PRINCIPLES ESTABLISHED IN GREAT BRITAIN.

\rule{4cm}{2pt}

\normalsize WITH COPIOUS TABLES OF DIMENSIONS, \&c.

\small ILLUSTRATED WITH

\large A SERIES OF THIRTY-NINE LARGE DRAUGHTS,

\small AND NUMEROUS SMALLER ENGRAVINGS.

\rule{4cm}{2pt}

\small THE SECOND EDITION,

\footnotesize WITH ADDITIONS;

\tiny INCLUDING 

\large AN ACCURATE DRAUGHT,

\footnotesize CONTAINING THE PLANS, ELEVATIONS, AND SECTIONS.

\tiny OF THE

\singlespacing
\small NEW METHODS OF FITTING THE STORE-ROOMS, \&c. BETWEEN THE GUN-DECK AND ORLOP OF 

SHIPS OF THE LINE, \&c.

\rule{8cm}{1pt}

\footnotesize LONDON:

\scriptsize PRINTED FOR STEEL AND CO. CHART-SELLERS TO THE ADMIRALITY, AT THEIR

\scriptsize NAVIGATION-WAREHOUSE, 70, CORNHILL.

\rule{3cm}{1pt}

\large 1812.
	\end{center}
\end{titlepage}
\restoregeometry

\part{BOOK THE FIRST. \\
\footnotesize EXPLAINING THE FIRST ELEMENTARY AND THEORETIC PRINCIPLES}
\setlength{\parindent}{10pt}
\chapter[EXPLANATION OF TERMS, \&C. USED IN SHIP-BUILDING]{CHAPTER 1. \\
\footnotesize \singlespacing AN EXPLANATION OF THE TERMS, AND SOME ELEMENTARY PRINCIPLES, REQUISITE TO BE UNDERSTOOD IN THE THEORY AND PRACTICE OF SHIP-BUILDING.}

\textbf{ABAFT.} The hinder part of a ship, or toward the stern.

\textbf{ABOARD.} Within, or upon, a ship. 

\textbf{ABREAST}. Alongside of, or opposite to; as in the case of two or more ships lying with their sides parallel, and their heads equally advanced. With regard to objects within the ship, this term implies on a line parallel with the beam, or at right angles with the ship's length; as "the \textsc{Fenders} should be placed abreast, or by the side of, the main hatchway."

\textbf{ABSCISSE}  See \textsc{Conic Sections}. 

\textbf{ACUTE ANGLE.} See \textsc{Angle}. This sort of angle upon wood, \&c. is by shipwrights denominated an under bevelling. See \textsc{Bevelling}. \textit{See also Circle, Fig. 2. Plate A.}

\textbf{AFLOAT.} Borne up by, or floating in, the water.

\textbf{AFORE.} The fore part of a ship, or toward the stem.

\textbf{AFT.} Towards, or near, the stern. 

\textbf{AFTER-BODY.} That part of a ship's body abaft the midships or dead-flat. (See \textsc{Bodies}.) This term is more particularly used in expressing the \textit{figure} or \textit{shape} of that part of the ship. See \textit{Body Plan, Plate 1.}

\textbf{AFTER PART OF THE SHIP.} All that part towards the stern, from the DEAD-FLAT, or broadest part of the ship . Or, with regard to the respective position of things placed in the direction of the ship's length, the term After denotes that which is nearest to the stern. 

\textbf{AFTER TIMBERS.} All those timbers abaft the MIDSHIPS OR DEAD FLAT. 

\textbf{AHEAD.} Any thing, which is situated on that point of the compass to which a ship's stem is directed, is said to be ahead of her. Objects on board are said to be taken ahead when removed towards the stem. 

\textbf{AIR FUNNEL.} A cavity framed between the sides of some timbers, to admit fresh air into the ship, and convey the foul air out of it. \textit{See Disposition, Plate 3}. 

\textbf{AMIDSHIPS.} In midships, or in the middle of the ship, either with regard to her length or breadth. Hence that timber or frame which has the greatest breadth and capacity in the ship is denominated the midship bend. See Dead-Flat. \textit{See also Sheer Draught, Plate 1}. 

\textbf{ANCHOR.} The instrument of iron, \&c. used, by means of a cable, to retain the ship in her station. 

\textbf{ANCHOR-LINING}. The short pieces of plank, or of board, fastened to the sides of the ship, or to stantions under the fore channel, to prevent the bill of the anchor from wounding the ship's side, when fishing the anchor. \textit{See Sheer Draught, Plate 1.} 

\textbf{To ANCHOR STOCK.} To work planks in a manner resembling the stocks of anchors, by fashioning them in a tapering form from the middle, and working or fixing them over each other, so that the broad or middle part of one plank shall be immediately above or below the butts or ends of two others. This method, as it occasions a great consumption of wood, is only used where particular strength is required, as in the Spirkettings under ports, \&c. 

\textbf{AN-END.} The position of any mast, \&c. when erected perpendicularly on the deck. The top masts are said to be AN - END when they are hoisted up to their usual stations. This is also a common phrase for expressing the driving of any thing in the direction of its length, as to force one plank, \&c. to meet the butt of another. 

\textbf{ANGLE.} A corner or point where two lines or two planes meet; as the lines \textbf{AB} and \textbf{CB}. An Angle is sometimes denoted by the single letter placed at the angular point, as \textbf{B}, or by three letters, of which that in the middle denotes the angle, as \textbf{ABC}. They are measured by the arch of a circle described from the centre with any radius, and are said to be greater or less according to the length of the arc con tained between the legs or sides. If an angle contains exactly 90 degrees, (the one - fourth of the number of degrees in which every circle is supposed to be divided,) it is formed by one line per pendicular to another, and is called a Right Angle, as \textbf{ABC} or \textbf{ABD}. If containing more than 90 degrees, it is said to be an OBTUSE - ANGLE, as \textbf{CBE}. If less than 90 degrees, an ACUTE - ANGLE, as \textbf{DBE}. An OBLIQUE - ANGLE is a common name for any angle that is not a right one, whether acute or obtuse. \textit{See Circle, Plate A.} 
	
	\textbf{ANGLE of DIRECTION.} That angle which is comprehended between the lines of direction of two conspiring forces; as of wind and tide. 

	\textbf{Angle of ELEVATION.} That angle which is comprehended between a line of direction and


\end{document}